\documentclass[11pt]{article}
\usepackage[utf8]{inputenc}

% add LaTeX packages to use here
\usepackage{amsmath}
\usepackage{amssymb}
\usepackage{amsfonts}
\usepackage{amsthm}
\usepackage{fancyhdr}
\usepackage{lastpage}
\usepackage{enumitem}
\usepackage{framed}
\usepackage[most]{tcolorbox}
\usepackage{geometry}
\usepackage{graphicx}

 % set dimensions for page layout
\geometry
{
 left=5em,
 right=5em,
 bottom=5em,
 top=6em,
 headheight=110pt,
 showframe=false
}

\setlist[itemize]{leftmargin=*} % prevents indenting of itemize

% abbreviations for some common math symbols
\newcommand{\Rset}{\hbox{$\mathbb R$}}
\newcommand{\Nset}{\hbox{$\mathbb N$}}
\newcommand{\Pset}{\hbox{$\mathbb{N}^{+}$}}
\newcommand{\Zset}{\hbox{$\mathbb N$}}
\newcommand{\Qset}{\hbox{$\mathbb Q$}}

% theorem style
\newtheoremstyle{thmstyle}% name of the style to be used
  {0pt}% measure of space to leave above the theorem. E.g.: 3pt
  {0pt}% measure of space to leave below the theorem. E.g.: 3pt
  {}% name of font to use in the body of the theorem
  {}% measure of space to indent
  {\bfseries}% name of head font
  {.}% punctuation between head and body
  { }% space after theorem head; " " = normal inter-word space
  {}% Manually specify head

% theorem environment instance
\theoremstyle{thmstyle}
\newtheorem{theorem}{Theorem}

% shaded and framed solution environment 
\makeatletter
\newenvironment{shadedSolutionBox}
  {\setlength{\OuterFrameSep}{0in}%
  \definecolor{shadecolor}{gray}{.8}% shading of shaded solution box
  \bigskip%
  \@nameuse{shaded*}\par\noindent\ignorespaces \textit{Solution}.}
  {\hspace{\stretch{1}}\rule{1.5ex}{1.5ex}% adds filled box 
  \@nameuse{endshaded*}%
  \bigskip}
\makeatother

% shaded and framed theorem environment 
\makeatletter
\newenvironment{thm}
  {\setlength{\OuterFrameSep}{0in}%
  \definecolor{shadecolor}{gray}{1}% shading of shaded Theorem box
  \@nameuse{snugshade*}\par\noindent\ignorespaces%
   \@nameuse{theorem}}
  {\hspace{\stretch{1}}\scalebox{1.5}{\hbox{$\triangleleft$}}% adds triangle shape
  \@nameuse{endtheorem}%
  \@nameuse{endsnugshade*}%
  }
\makeatother

% header and footer elements of every page except the first.
\pagestyle{fancy}
\fancyfoot[L]{\textsc{CISC {\small\selectfont 3230}} }
\fancyhead[R]{{\small\selectfont\textsc{\studentLastName}}}
\fancyfoot[C]{{\small\selectfont\assignmentName}}
\fancyfoot[R]{{\small\selectfont\thepage\ of \pageref{LastPage}}}
\renewcommand{\headrulewidth}{0.8pt}
\renewcommand{\footrulewidth}{0.4pt}

% hline with variable thickness
\makeatletter
\def\thickhline{%
  \noalign{\ifnum0=`}\fi\hrule \@height \thickarrayrulewidth \futurelet
   \reserved@a\@xthickhline}
\def\@xthickhline{\ifx\reserved@a\thickhline
               \vskip\doublerulesep
               \vskip-\thickarrayrulewidth
             \fi
      \ifnum0=`{\fi}}
\makeatother

% length instance for \thickhline
\newlength{\thickarrayrulewidth} 
\setlength{\thickarrayrulewidth}{.8pt}

% header and footer for first page
\fancypagestyle{firstpage}
{
\fancyhf{}
\renewcommand{\footrulewidth}{0.4pt}
\renewcommand{\headrulewidth}{0pt}
\fancyhead[C]{%
\begin{tabular*}{\textwidth}{@{\extracolsep{\fill}}@{}l @{} c @{} r @{} }
{\small\selectfont\courseName}&{\normalsize\selectfont\assignmentName}&{\small\selectfont\studentFirstName\ \studentLastName}\\
\thickhline
&&{\scriptsize\selectfont\collaboratorNames}
\end{tabular*}%
}
\fancyfoot[R]{{\small\selectfont\thepage\ of \pageref{LastPage}}}
\fancyfoot[L]{{\footnotesize\selectfont\pdfcreationdate}}
}

\newcommand{\courseName}{Theoretical Computer Science} % course name

% your first name, your last name, and the assignment name
\newcommand{\studentLastName}{Nikabadze} % your last name
\newcommand{\studentFirstName}{Luka} % your first name (and middle name, if applicable)
\newcommand{\assignmentName}{Homework 7} % the assignment name
\newcommand{\collaboratorNames}{[First Name Initial]. Last Name} % if you worked with anyone to complete any part of the assignment, include the initial of the first name (and middle name, if applicable) and full last name of each of your collaborators, separated by commas (e.g., if you worked with Arthur Paul Pedersen and Sandra Lee, include "A.P. Pedersen, S. Lee")




\begin{document} % marks the beginning of the document

\thispagestyle{firstpage} % institutes page style for first page

\setlength{\abovedisplayskip}{20pt} % space above math in align* environment
\setlength{\belowdisplayskip}{20pt} % space below math in align* environment



% marks the beginning of the document body


\begin{itemize}\setlength{\itemsep}{1em} % \itemsep is the spacing between items in environment
\item[2.9]  
A language can be split into two languages which are defined as follows,

$A_1 = {a^i b^j c^k} | i,j,k \geq 0, i = j$  and 
$A_2 = {a^i b^j c^k} | i,j,k \geq 0, j = k$

using the language A1 and A2 the users can construct a CFG for $A_1$ and $A_2$
the grammar for language A is the union of grammar of two languages which is defined as follows  $S \rightarrow  S_1 | S_2$

in the language $A_1$ the values of i and j are equal so there must be equal number of a's and b's  in the language $A^1$
CFG for language  $A^1$ and $A^2$ are given below

$S^2 \rightarrow aS^2 | F | \in $

$F \rightarrow bFc | \in$

since the generating string $w= a b c $ using the language a, either $s_1$ or $s_2$ can be user.
therefore the context free grammar for the language A is ambiguous.

\item[2.10]  

As we know from previous example we have $A_1 and A_2$ 
$A_1 = {a^i b^j c^k} | i,j,k \geq 0, i = j$  and 
$A_2 = {a^i b^j c^k} | i,j,k \geq 0, j = k$

push down automation follows as 

Read and push $a's$

read $b'$ while popping $a's$

if $b's$ finish when stack is empty skip $c's$ on input and accept.

\item[2.14]

lets add a new start variable $S_0$ and rule $S_0 \rightarrow A.$
so grammar is

$S_0 \rightarrow A$

A $S_0 BAB | B | \epsilon$

B $\rightarrow 00 | \epsilon$

now we remove rules that contain $\epsilon$
$S_0 \rightarrow A | \epsilon$

$A \rightarrow BAB | BA | AB | A | B | BB$

$B \rightarrow 00$

the rule $S_0 \rightarrow \epsilon$ is accepted since $S_0$ is the start variable and that is allowed in Chomsky normal form.

now remove the unit rules

$S_0 \rightarrow A | \epsilon$

$A \rightarrow BAB | BA | AB | 00 | BB$

$B \rightarrow 00 $

$S_0 A \epsilon$

$A \rightarrow BAB | BA | AB | 00 | BB$

$B \rightarrow 00$

$S_0 \rightarrow BAB| BA | AB | 00 | BB | \epsilon$

$a \rightarrow BAB | BA | AB | 00 |BB$

$B \rightarrow 00$

now we replace third placed terminals 0 by variable U with new.

$S_0 \rightarrow BAB | BA | AB | UU | BB | \epsilon$

$A \rightarrow BAB| BA | AB | UU | BB$

$B \rightarrow UU$

$U \rightarrow 0$

$A_1 \rightarrow AB$

This is the final CFG in Chomsky normal form equivalent to the given CFG.

\item[2.26]
Given that G is a CFG in Chomsky normal form. the length of the string w $\in$ L(G) is n$ \geq$ 1 for the string w. it is required to show that exactly $2n-1$ steps are required for the derivation of string w. it can be proved applying the induction method by on the string w of length n.

for $n = 1$ consider string " a " of length 1 in Chomsky normal form the valid derivation for this will be $s \rightarrow a$. the number of steps can be obtained as follows 

the number of steps can be obtained as follows 
$2n - 1 = 2(1)-1 = 2-1 = 1$

now $n = k+1 $ is in chomsky normal form.

since $n>1$ consider a language in CNF where derivation starts with start symbol S.

$S \rightarrow BC$

$B \rightarrow *x$

$C \rightarrow *y$

using the inductive hypothesis, for the above language in CNF the length of any derivation of string w must be.

$1+2(|x|-1)+(2|y|-1)=2|x|+2|y|+1-1-1=2(|x|+|y|)-1$

here $n = |x|+|Y|$

since $B \rightarrow *x$ has a length of |x| and $C \rightarrow *y$ has a length of |y|.

hence it is proved that it requires 2n-1 steps required for the derivation of string  $w \in L(G)$ in Chomsky normal form.
 
\end{itemize}



\end{document}